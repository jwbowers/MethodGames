\documentclass[12pt]{article}
\usepackage{parskip}
\usepackage{natbib}
\usepackage{amsmath}
\usepackage{url}
\usepackage[pdftex,colorlinks=true,citecolor=black,raiselinks=false]{hyperref}
\usepackage{tgtermes}
\usepackage[T1]{fontenc}

\title{Method Games:\\ Assessing Methods for Discovery}
\author{Jake Bowers}
\date{\today}

\begin{document}
\maketitle

Imagine assessing a promising method for pattern discovery using a game.  One
scholar would invent a true pattern of features, generate an outcome and
perhaps hide this pattern amid irrelevant information.  For example, the game
designer might provide 15 binary features of 40 cases to the players. Players
would compete to discover the hidden truth.  One version of the method game
would require that participants use a particular algorithm. A second version
would allow participants to choose their own algorithm. For example, some
might choose a QCA variant \citep{rihoux2008configurational}, others would
implement an adaptive lasso \citep{zou2006adaptive} and still others might
prefer one of the many competitors to the lasso, such as the smoothly clipped
absolute deviation (SCAD) penalty \citep{fan2001variable}, random forests
~\citep{breiman2001random}, or kernel-regularized least squares
~\citep{hainmueller2012kernel}.)\footnote{\citet{hasttibfried09} provide an
  excellent overview of many of the techniques known as ``machine learning''
  or ``data mining''. The adaptive lasso and SCAD techniques used in this
  article belong to a family of machine learning algorithms known as penalized
  linear models. By setting coefficients to zero rather than privileging fit
  to data,  these procedures fit linear models to outcomes and aim to return
  coefficients of zero for irrelevant features, thereby revealing relevant
  features. These algorithms choose coefficients $\beta_1, \ldots, \beta_P$
  for $P$ features (and arbitrary combinations thereof), $X_{i1}, \ldots,
  X_{iP}$, related to some outcome, $y_i$, to minimize a function of the sum
  of squared prediction error (i.e. least squares) plus a penalty function
  that rewards solutions with some collection of $\beta_p$ set to 0. The
  objective function tends to look like $\sum_{i=1}^N (y_i - ( \beta_0 +
  \beta_1 X_{i1} + \ldots + \beta_P X_{iP}) )^2 + \sum_{p=1}^P
  p(\lambda,\beta_p)$.  The tuning parameter, $\lambda$, determines the
  relative importance of the penalty function in the objective function
  compared to the least squares function.  The adaptive lasso penalty is
  $p(\lambda,\beta_p,w_p)=\lambda  w_p|\beta_p|$ where $w_p=1/\hat{\beta}_p$
  and $\hat{\beta}_p$ arises from a previous linear model (here a ridge
  regression but often an OLS regression). The SCAD replaces the lasso penalty
  with a function designed to have no penalty when $\beta_p=0$ (like the
  adaptive lasso) but then to rise smoothly to penalize very large $\beta_p$
  at a decreasing rate:  $p(\lambda,\beta_p,a)=\begin{cases} \lambda
    |\beta_p|, & \text{if } |\beta_p|\le \lambda; \\ - \left(
      \frac{|\beta_p|^2 - 2 a \lambda |\beta_p| + \lambda^2}{2 (a-1)} \right),
    & \text{if } \lambda < |\beta_p| \le a \lambda; \\
    \frac{(a+1)\lambda^2}{2},  & \text{if } |\beta_p| > a \lambda
  \end{cases}$, where $a > 2$ and $\lambda > 0$.  Some see adaptive lasso as
  an approximation or competitor to the SCAD penalty \cite[page
  92]{hasttibfried09}. }
 
%%   \citet{fan2001variable} define the penalty by its derivative
%%   such that if the penalty function for a given $\beta$ is $p_\lambda(\beta)$ then
%%   $p'_{\lambda}(\beta)=\lambda \left\{ I(\beta \le \lambda) + \frac{ (a
%%       \lambda - \beta)_{+} }{ (a-1) \lambda} I(\beta > \lambda) \right\}$.    
%% }
%% 

In the first version of the competition, we would learn about craft: in
different hands the same method may perform differently. The results of this
competition would teach us about the judgment required to use the method
successfully.  In the version of the game where players choose different
approaches, we could learn how different methods compare in their ability to
address a given problem.\footnote{Although we might also confuse learning
  about method with a discovery that some researchers have excellent judgment
  and luck.}

If, however, time were short or players difficult to recruit, one could
approximate such a game by building on \cite{lucasfk2014}. That is, a single
scholar could generate a true relationship as if kicking off a real method
game but then write a computer program to compare the effectiveness of
different algorithms. Imagine that a scholarly literature focusing on 40 cases
suggests that a complex dependent pattern of binary features $X_{i1}, \ldots,
X_{iP}$, of a given case $i$, drives outcomes (say, for $P=5$, $Y_i= \left\{
  (X_{i1} \cdot X_{i2} \cdot X_{i3} ) \text{ OR } ( X_{i4} \cdot X_{i5})
\right\}$ all $X_{ip}  \in \{0,1\}$ and thus $Y_i \in \{0,1\}$).  Further,
imagine that three methods suggest themselves as useful a priori: (1) QCA, (2)
the adaptive lasso and (3) iterative sure independence screening with a SCAD
plugin (ISIS/SCAD) \citep{fan2008sure}.  \citet{fan2001variable} proved that
the SCAD penalty would correctly set false parameters to 0 as $n \rightarrow
\infty$ given a reasonable choice of tuning parameters in contrast to the
simple lasso proposed by \cite{tibshirani1996regression} --- that is, SCAD has
an oracle property but the simple lasso does not. Later,
\citet{zou2006adaptive} showed a modification of the lasso penalty (the
adaptive lasso) does have an oracle property given well-chosen tuning
parameters and weights.  And, \citet{fan2008sure} demonstrate that, when the
number of irrelevant features is much larger than the number of cases (for
example, when each case has 10,000 measured features but we only observe 50
cases), a preliminary dimension reduction step (ISIS) improves the performance
of the SCAD penalty.  Although QCA does not promise to find the truth as
information increases, it appears well suited to discovering complex
comparisons and it does not require tuning parameters. This essay presents the
results from a script that implements a machine version of the method game to
compare QCA, the adaptive lasso, and ISIS/SCAD.\footnote{Interested readers
  can download the code from \url{https://github.com/jwbowers/MethodGames}.}

Machine players are naive. Assessing the performance of a machine will not
tell us about the craft by which human scholars exploit a method.  Further,
any single collection of case attributes can idiosyncratically advantage one
method over another. For fairness, and to approximate the kind of natural
variation one would see if different human players were involved in the game,
the script generated a different set of features for each machine player
although the outcomes arose from the same deterministic true
function as described above.  This competition involved 800 players each using
all three approaches to seek the truth.  The script runs two contests. The
easier of the two games presents players with a five column dataset: each
column represents a part of the truth, and players focus on finding the true
combinations of the existing features.  The hard game differs from the easy
game only in that the data set contains 10 irrelevant case features in
addition the original 5.  The script counts a player as successful if it found
the truth and only the truth. In the easy game, QCA, the adaptive lasso, and
ISIS/SCAD found the truth for 18\%, 82\%, and 96\% of the players
respectively. In the hard game, QCA, the adaptive lasso, and ISIS/SCAD found
the truth for 0\%, 33\%, and 61\% of the players respectively.

These results do not suggest that we should discard QCA and the adaptive lasso
in favor of ISIS/SCAD.  A large and growing literature both criticizes and
builds on the adaptive lasso. For example, if the features are highly
inter-dependent, we might prefer adaptive versions of the fused lasso
\citep{rinaldo2009properties}, the grouped lasso \citep{wang2008note}, or the
elastic net \citep{ghosh2011grouped, zou2004regression}. And future
methodology building on the simple QCA might adapt insights from machine
learning to overcome current shortcomings or advise against the use of QCA for
particular designs and data.  Most scholars would prefer a technique that
recovers the truth more than 60\% of the time, so one might use these results
to motivate work to improve the performance of the ISIS/SCAD or to find a
substitute. Obviously, a real competition with skilled human players with
excellent judgment might have produced different results.  After all, those
who investigate and modify the script will notice little expertise and craft
in use of the techniques: for example, the tuning parameters for the adaptive
lasso were chosen fairly naively, only one open-source implementation of QCA
was used, and many other small but potentially consequential decisions appear
throughout.

Fruitful communication about methods involves comparing the successes of
different methods in the hands of different human scholars confronting
specific research designs, theoretical goals, and existing observations. The
method game proposed here would help us learn not only about a method in
an abstract sense, but about the craft of using said method in comparison with
other methods. A machine version of the method game enables a fast and cheap
and controlled way to begin to build a comparative understanding of the
methods and/or to motivate people to engage in a real method game.



\bibliographystyle{apsr}
\bibliography{/Users/jwbowers/Documents/PROJECTS/BIB/big}
\end{document}


[^2]: 
